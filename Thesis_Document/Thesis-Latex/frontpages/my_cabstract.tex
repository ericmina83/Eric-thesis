  在本研究中,我們針對遊戲過程中的遊玩特徵 (gameplay patterns) 進行抽象化,使用程序化生成技術產生帶有意義遊戲關卡內容,藉此消彌或降低因隨機性所產生的不穩定要素,以改善並豐富遊戲體驗,最終提供一完整的遊戲關卡生成解決方案。

我們將「Mission/Space 框架」與「Multi-segment 演化」兩種關卡生成方法結合並修改予以適之,保留了前者追求的遊戲進程之順序性,後者帶來穩定且多樣化的遊戲內容,希冀藉此提升整體遊戲體驗、相輔相成。透過將遊戲關卡的劃分為任務 (Missions) 與空間 (Spaces) 兩種結構後,空間會依照任務結構進行有意義的同構轉換,並依照遊玩特徵定義基因演算法 (Genetic Algorithms) 的適應性函數,空間將透過基因演算法演化得出最適的遊戲物件佈局。

利用此關卡生成解決方案來設計關卡能有效減少開發時間、容易掌握遊戲各元件的配置外,同時讓玩家在進行遊戲時能夠遵循關卡設計師的劇情脈絡,亦能夠體驗到有意義且多樣化的遊戲關卡內容。

\textbf{關鍵字}:遊戲關卡生成、程序化內容生成、遊玩特徵、基因演算法