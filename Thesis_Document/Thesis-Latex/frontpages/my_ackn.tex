完成本研究與碩士學位,特別感謝我的指導教授戴文凱老師,總能給予我們研究生莫大的援助與鼓勵。在我們遇到研究瓶頸時,適時地為我們指點迷津,大家總是對於自己身為 GAME Lab 的一份子、身為老師的學生感到萬分欣喜與歡欣,望老師帶領的 GAME Lab 轉移至台科後,仍能夠像東華時一樣風采,甚至青出於藍。

感謝中研院資訊所的陳昇瑋老師與 MMNET 的各位夥伴,懷中、之凡、Erin、章魚哥、Wendy、Helen、浩軒、Jason、Zack、書庭、義銘、宗榮、Emily、凱翔、俊甫、冠豪、宗導、宜蓓、旻君、柏亨、鈞閔。諸位對我過去的照顧與指教,讓我鼓足勇氣前往台科攻讀研究所,在中研院的近兩年的實習時光是我最美好的回憶之一。

感謝台科 GAME Lab 的夥伴們,承翰、熙庭、柏勛、Bagus、廣柏、品陵、濬安、允斌、廸耀、小葉、來來、秝安、宇哲,我們一起學習、做研究、辦研討會、出遊放鬆,希望大家不論是未來的學業或工作都能順順利利。感謝創動團隊的蔡建毅先生與其團隊成員,提供我們相當寶貴的開發經驗與素材支援,讓我們在短時間內汲取到業界的專業知識。我還要感謝產學計劃的夥伴們,品陵、廣柏、小峰、定豪、浩辰、韋誠、成峰、敏宣、公耀、如萲、鈺哲,因為你們的扶持與配合,希冀未來的學習之路上諸位皆能一帆風順。以及感謝台大的凱翔、柏任與台科工工的均揚,讓我們共同完成了課堂的作業與專題,望各位畢業之後的工作皆能得心應手。

最後感謝我的家人們,讓我無後顧之憂並能專心於學習與研究工作之上,祝福爸爸、媽媽與哥哥,能夠開心地度過每一天。