% Quickly create the table.
\newcommand{\gasettingstable}[4]{{
\begin{table}[H]
  \centering
  \caption{#1}
  \label{#2}
  \bigskip
  \vspace{-5mm}
  \begin{minipage}[t]{0.48\linewidth}
    \begin{tabular}[t]{| c | c | c |}
      \hline
      \multicolumn{1}{ |c| }{回合次數}
        & \multicolumn{1}{ c| }{世代數量}
        & \multicolumn{1}{ c| }{個體數量} \\\hline
      #3
      \hline
    \end{tabular}
  \end{minipage}
  \begin{minipage}[t]{0.48\linewidth}
    \begin{tabular}[t]{| c | c | c |}
      \hline
      \multicolumn{1}{ |c| }{指標}
        & \multicolumn{1}{ c| }{權重}
        & \multicolumn{1}{ c| }{備注} \\\hline
      #4
      \hline
    \end{tabular}
  \end{minipage}
\end{table}
}}


\chapter{實驗結果與分析}
\label{cha:experiment}

本章節將探討前述章節之方法是否符合研究目標,將進行以下實驗:第~\ref{sec:experiment-results}小節中,針對所挑選的房間採用不同權重的適應性函數,會如何影響房間內的遊戲物件之配置結果。第~\ref{sec:experiment-yyy} 小節中,觀察染色體中的基因數量對於演化過程之影響。

\section{實驗定義}
\label{sec:experiment-definition}

我們使用 Invector 公司開發的 \textit{Third Person Controller - Melee Combat Template} 套件(以下簡稱 \textit{3rdPC})進行遊戲物件的設置。\textit{3rdPC} 提供遊戲開發者快速建構玩家控制角色 (Player Character) 、非玩家控制角色 (Non-Player Character) 與其角色控制器或人工智能,並可製作任何類型的第三人稱動作冒險遊戲或角色扮演遊戲。第~\ref{ssec:experiment-gameplaymanual} 小節中,說明 \textit{3rdPC} 提供的玩家角色的操作使用說明;第~\ref{ssec:experiment-gameobjects} 小節中,解釋實驗中的採用遊戲物件,其象徵意義與玩家角色互動方式;第~\ref{ssec:experiment-parameters} 小節中,將定義屬實驗變因的相關參數設定,包含任務語法與基因演算法。

\subsection{遊戲操作說明}
\label{ssec:experiment-gameplaymanual}

玩家所控制的角色能夠移動、攻擊、防禦、跳躍與蹲下等動作。移動使用方向鍵,角色朝前後左右與其斜向組合鍵的方向移動,過程中按下 shift 按鍵將會切換至奔跑模式;攻擊使用滑鼠左鍵單擊,連續使用能夠進行三階段的攻擊,每次攻擊將消耗耐力值;防禦使用滑鼠右鍵單擊,能夠減少受到敵方攻擊的損傷;跳躍按鍵鍵盤 space 按鍵,角色能夠進行原地跳躍,在移動或奔跑的過程中亦可使用;蹲下按下鍵盤 c 按鍵,角色能夠緩慢的移動,適合觀察敵方戰況的同時進行移動。

\subsection{遊戲物件說明}
\label{ssec:experiment-gameobjects}

(編輯中)

\subsubsection{敵人}
\label{sssec:experiment-gameobjects-enemy}

敵人 (編輯中)。

\subsubsection{寶箱}
\label{sssec:experiment-gameobjects-treasure}

當玩家靠近並朝向寶箱時,按下鍵盤 E 按鍵便可以開啟寶箱,取得寶箱內容物。

\subsubsection{陷阱}
\label{sssec:experiment-gameobjects-trap}

當玩家碰觸到陷阱時,會受到一定程度的損傷。

\subsection{實驗參數設定}
\label{ssec:experiment-parameters}

在任務語法階段時,我們將非合法 ... (編輯中)

房間容器的遊戲物件佈局演化階段時,每一次世代的演化過程有 $80\%$ 機率於父母代進行兩點交配 (two-point crossover);$10\%$ 機率令衍生子代會進行突變,個體的染色體有 $5\%$ 至 $20\%$ 的基因會轉換成其它的遊戲物件種類。

\section{資料收集}
\label{sec:experiment-datacollection}

在資料收集階段中,將收集第~\ref{sec:method-segments} 節中,房間容器進行基因演算法演化時,於實驗過程中輸出 CSV 格式資料。表~\ref{tbl:structure-of-rawdata-scores} 紀錄各回合、世代中,其各個個體(單一染色體)的適應值的得分狀況;表~\ref{tbl:structure-of-rawdata-positions} 紀錄各回合、世代中,其各個個體(單一染色體)的所有座標資訊與其遊戲物件類型。更多的原始資料節錄內容可參見附錄資源。

\begin{table}[!htb]
  \centering
  \caption{演化適應值資料節錄}
  \label{tbl:structure-of-rawdata-scores}
  \bigskip
  \begin{tabular}{| c | c | c | l | l |}
    \hline
    \multicolumn{1}{ |c| }{回合編號}
      & \multicolumn{1}{ c| }{世代編號}
      & \multicolumn{1}{ c| }{染色體編號}
      & \multicolumn{1}{ c| }{指標}
      & \multicolumn{1}{ c| }{得分} \\\hline
    1 & 1 & 1 & Block     & 0 \\
    1 & 1 & 1 & Intercept & 0 \\
    1 & 1 & 1 & Patrol    & 0 \\
    1 & 1 & 1 & Guard     & 0 \\
    1 & 1 & 1 & Support   & 0 \\
    1 & 1 & 1 & Block     & 0 \\
    1 & 1 & 1 & Intercept & 0 \\
    \hline
  \end{tabular}
\end{table}

\begin{table}[!htb]
  \centering
  \caption{演化座標資料節錄}
  \label{tbl:structure-of-rawdata-positions}
  \bigskip
  \begin{tabular}{| c | c | c | l | c |}
    \hline
    \multicolumn{1}{ |c| }{回合編號}
      & \multicolumn{1}{ c| }{世代編號}
      & \multicolumn{1}{ c| }{染色體編號}
      & \multicolumn{1}{ c| }{座標}
      & \multicolumn{1}{ c| }{遊戲物件類型} \\\hline
    1 & 1 & 1 & (1.0, 1.0, 0.0) & Empty \\
    1 & 1 & 1 & (0.0, 1.0, 0.0) & Empty \\
    1 & 1 & 1 & (0.0, 1.0, 1.0) & Empty \\
    1 & 1 & 1 & (0.0, 1.0, 2.0) & Empty \\
    1 & 1 & 1 & (0.0, 1.0, 3.0) & Empty \\
    \hline
  \end{tabular}
\end{table}

\section{演化結果與其品質}
\label{sec:experiment-results}

在第~\ref{ssec:method-segments-appliedonvolumes} 小節中,展示了寶藏房與戰鬥通道(狹路驅逐、鎮守要道)三種空間的局部佈局演化結果,隨著房間容器搭配不同的適應性函數,便能夠生成出多樣性的遊戲物件佈局。在本小節中,將針對上述三種房間容器的演化結果做更進一步的分析。

\subsection{寶藏房}
\label{ssec:experiment-results-treasure}

\subsubsection{實驗 A - 世代數量的影響}
\label{sssec:experiment-results-treasure-gen}

A 實驗內文實驗內文實驗內文實驗內文實驗內文實驗內文實驗內文實驗內文實驗內文實驗內文實驗內文實驗內文實驗內文實驗內文實驗內文實驗內文實驗內文實驗內文實驗內文實驗內文實驗內文實驗內文實驗內文實驗內文實驗內文實驗內文實驗內文實驗內文實驗內文實驗內文

\gasettingstable{實驗 A1 之基因演算法參數配置}
  {tbl:settings-of-experiment-results-treasure-gen-i}
  { $1$ & $50$ & $100$ \\ }
  {
    守衛點       & $1.00$ & \\
    遊戲物件數量 & $1.00$ & $Limit: [2, 5]$ \\
  }

A 實驗內文實驗內文實驗內文實驗內文實驗內文實驗內文實驗內文實驗內文實驗內文實驗內文實驗內文實驗內文實驗內文實驗內文實驗內文實驗內文實驗內文實驗內文實驗內文實驗內文實驗內文實驗內文實驗內文實驗內文實驗內文實驗內文實驗內文實驗內文實驗內文實驗內文

\gasettingstable{實驗 A2 之基因演算法參數配置}
  {tbl:settings-of-experiment-results-treasure-gen-ii}
  { $1$ & $100$ & $100$ \\ }
  {
    守衛點       & $1.00$ & \\
    遊戲物件數量 & $1.00$ & $Limit: [2, 5]$ \\
  }

A 實驗內文實驗內文實驗內文實驗內文實驗內文實驗內文實驗內文實驗內文實驗內文實驗內文實驗內文實驗內文實驗內文實驗內文實驗內文實驗內文實驗內文實驗內文實驗內文實驗內文實驗內文實驗內文實驗內文實驗內文實驗內文實驗內文實驗內文實驗內文實驗內文實驗內文

\gasettingstable{實驗 A3 之基因演算法參數配置}
  {tbl:settings-of-experiment-results-treasure-gen-iii}
  { $1$ & $200$ & $100$ \\ }
  {
    守衛點       & $1.00$ & \\
    遊戲物件數量 & $1.00$ & $Limit: [2, 5]$ \\
  }

\subsubsection{實驗 B - 個體數量的影響}
\label{sssec:experiment-results-treasure-idv}

A 實驗內文實驗內文實驗內文實驗內文實驗內文實驗內文實驗內文實驗內文實驗內文實驗內文實驗內文實驗內文實驗內文實驗內文實驗內文實驗內文實驗內文實驗內文實驗內文實驗內文實驗內文實驗內文實驗內文實驗內文實驗內文實驗內文實驗內文實驗內文實驗內文實驗內文

\gasettingstable{實驗 B1 之基因演算法參數配置}
  {tbl:settings-of-experiment-results-treasure-idv-i}
  { $1$ & $100$ & $50$ \\ }
  {
    守衛點       & $1.00$ & \\
    遊戲物件數量 & $1.00$ & $Limit: [2, 5]$ \\
  }

B 實驗內文實驗內文實驗內文實驗內文實驗內文實驗內文實驗內文實驗內文實驗內文實驗內文實驗內文實驗內文實驗內文實驗內文實驗內文實驗內文實驗內文實驗內文實驗內文實驗內文實驗內文實驗內文實驗內文實驗內文實驗內文實驗內文實驗內文實驗內文實驗內文實驗內文

\gasettingstable{實驗 B2 之基因演算法參數配置}
  {tbl:settings-of-experiment-results-treasure-idv-ii}
  { $1$ & $100$ & $100$ \\ }
  {
    守衛點       & $1.00$ & \\
    遊戲物件數量 & $1.00$ & $Limit: [2, 5]$ \\
  }

B 實驗內文實驗內文實驗內文實驗內文實驗內文實驗內文實驗內文實驗內文實驗內文實驗內文實驗內文實驗內文實驗內文實驗內文實驗內文實驗內文實驗內文實驗內文實驗內文實驗內文實驗內文實驗內文實驗內文實驗內文實驗內文實驗內文實驗內文實驗內文實驗內文實驗內文

\gasettingstable{實驗 B3 之基因演算法參數配置}
  {tbl:settings-of-experiment-results-treasure-idv-iii}
  { $1$ & $100$ & $200$ \\ }
  {
    守衛點       & $1.00$ & \\
    遊戲物件數量 & $1.00$ & $Limit: [2, 5]$ \\
  }

\subsubsection{小結}
\label{sssec:experiment-results-treasure-summary}

實驗小結實驗小結實驗小結實驗小結實驗小結實驗小結實驗小結實驗小結實驗小結實驗小結實驗小結實驗小結實驗小結實驗小結實驗小結實驗小結





\subsection{戰鬥通道(狹路驅逐)}
\label{ssec:experiment-results-narrow}

\subsubsection{實驗 C - 世代數量的影響}
\label{sssec:experiment-results-narrow-gen}

C 實驗內文實驗內文實驗內文實驗內文實驗內文實驗內文實驗內文實驗內文實驗內文實驗內文實驗內文實驗內文實驗內文實驗內文實驗內文實驗內文實驗內文實驗內文實驗內文實驗內文實驗內文實驗內文實驗內文實驗內文實驗內文實驗內文實驗內文實驗內文實驗內文實驗內文

\gasettingstable{實驗 C1 之基因演算法參數配置}
  {tbl:settings-of-experiment-results-narrow-gen-i}
  { $1$ & $50$ & $100$ \\ }
  {
    阻攔點       & $1.00$ & \\
    攔截點       & $0.75$ & \\
    陷阱點       & $0.75$ & \\
    遊戲物件數量 & $1.00$ & $Limit: [2, 3]$ \\
  }

C 實驗內文實驗內文實驗內文實驗內文實驗內文實驗內文實驗內文實驗內文實驗內文實驗內文實驗內文實驗內文實驗內文實驗內文實驗內文實驗內文實驗內文實驗內文實驗內文實驗內文實驗內文實驗內文實驗內文實驗內文實驗內文實驗內文實驗內文實驗內文實驗內文實驗內文

\gasettingstable{實驗 C2 之基因演算法參數配置}
  {tbl:settings-of-experiment-results-narrow-gen-ii}
  { $1$ & $100$ & $100$ \\ }
  {
    阻攔點       & $1.00$ & \\
    攔截點       & $0.75$ & \\
    陷阱點       & $0.75$ & \\
    遊戲物件數量 & $1.00$ & $Limit: [2, 3]$ \\
  }

C 實驗內文實驗內文實驗內文實驗內文實驗內文實驗內文實驗內文實驗內文實驗內文實驗內文實驗內文實驗內文實驗內文實驗內文實驗內文實驗內文實驗內文實驗內文實驗內文實驗內文實驗內文實驗內文實驗內文實驗內文實驗內文實驗內文實驗內文實驗內文實驗內文實驗內文

\gasettingstable{實驗 C3 之基因演算法參數配置}
  {tbl:settings-of-experiment-results-narrow-gen-iii}
  { $1$ & $200$ & $100$ \\ }
  {
    阻攔點       & $1.00$ & \\
    攔截點       & $0.75$ & \\
    陷阱點       & $0.75$ & \\
    遊戲物件數量 & $1.00$ & $Limit: [2, 3]$ \\
  }

\subsubsection{實驗 D - 個體數量的影響}
\label{sssec:experiment-results-narrow-idv}

D 實驗內文實驗內文實驗內文實驗內文實驗內文實驗內文實驗內文實驗內文實驗內文實驗內文實驗內文實驗內文實驗內文實驗內文實驗內文實驗內文實驗內文實驗內文實驗內文實驗內文實驗內文實驗內文實驗內文實驗內文實驗內文實驗內文實驗內文實驗內文實驗內文實驗內文

\gasettingstable{實驗 D1 之基因演算法參數配置}
  {tbl:settings-of-experiment-results-narrow-idv-i}
  { $1$ & $100$ & $50$ \\ }
  {
    阻攔點       & $1.00$ & \\
    攔截點       & $0.75$ & \\
    陷阱點       & $0.75$ & \\
    遊戲物件數量 & $1.00$ & $Limit: [2, 3]$ \\
  }

D 實驗內文實驗內文實驗內文實驗內文實驗內文實驗內文實驗內文實驗內文實驗內文實驗內文實驗內文實驗內文實驗內文實驗內文實驗內文實驗內文實驗內文實驗內文實驗內文實驗內文實驗內文實驗內文實驗內文實驗內文實驗內文實驗內文實驗內文實驗內文實驗內文實驗內文

\gasettingstable{實驗 D2 之基因演算法參數配置}
  {tbl:settings-of-experiment-results-narrow-idv-ii}
  { $1$ & $100$ & $100$ \\ }
  {
    阻攔點       & $1.00$ & \\
    攔截點       & $0.75$ & \\
    陷阱點       & $0.75$ & \\
    遊戲物件數量 & $1.00$ & $Limit: [2, 3]$ \\
  }

D 實驗內文實驗內文實驗內文實驗內文實驗內文實驗內文實驗內文實驗內文實驗內文實驗內文實驗內文實驗內文實驗內文實驗內文實驗內文實驗內文實驗內文實驗內文實驗內文實驗內文實驗內文實驗內文實驗內文實驗內文實驗內文實驗內文實驗內文實驗內文實驗內文實驗內文

\gasettingstable{實驗 D3 之基因演算法參數配置}
  {tbl:settings-of-experiment-results-narrow-idv-iii}
  { $1$ & $100$ & $200$ \\ }
  {
    阻攔點       & $1.00$ & \\
    攔截點       & $0.75$ & \\
    陷阱點       & $0.75$ & \\
    遊戲物件數量 & $1.00$ & $Limit: [2, 3]$ \\
  }

\subsubsection{小結}
\label{sssec:experiment-results-narrow-summary}

實驗小結實驗小結實驗小結實驗小結實驗小結實驗小結實驗小結實驗小結實驗小結實驗小結實驗小結實驗小結實驗小結實驗小結實驗小結實驗小結





\subsection{戰鬥通道(鎮守要道)}
\label{ssec:experiment-results-trunk}

\subsubsection{實驗 E - 世代數量的影響}
\label{sssec:experiment-results-trunk-gen}

E 實驗內文實驗內文實驗內文實驗內文實驗內文實驗內文實驗內文實驗內文實驗內文實驗內文實驗內文實驗內文實驗內文實驗內文實驗內文實驗內文實驗內文實驗內文實驗內文實驗內文實驗內文實驗內文實驗內文實驗內文實驗內文實驗內文實驗內文實驗內文實驗內文實驗內文

\gasettingstable{實驗 E1 之基因演算法參數配置}
  {tbl:settings-of-experiment-results-trunk-gen-i}
  { $1$ & $50$ & $100$ \\ }
  {
    阻攔點       & $1.00$ & \\
    支援點       & $0.75$ & \\
    巡邏點       & $0.50$ & \\
    遊戲物件數量 & $1.00$ & $Limit: [3, 5]$ \\
  }

E 實驗內文實驗內文實驗內文實驗內文實驗內文實驗內文實驗內文實驗內文實驗內文實驗內文實驗內文實驗內文實驗內文實驗內文實驗內文實驗內文實驗內文實驗內文實驗內文實驗內文實驗內文實驗內文實驗內文實驗內文實驗內文實驗內文實驗內文實驗內文實驗內文實驗內文

\gasettingstable{實驗 E2 之基因演算法參數配置}
  {tbl:settings-of-experiment-results-trunk-gen-ii}
  { $1$ & $100$ & $100$ \\ }
  {
    阻攔點       & $1.00$ & \\
    支援點       & $0.75$ & \\
    巡邏點       & $0.50$ & \\
    遊戲物件數量 & $1.00$ & $Limit: [3, 5]$ \\
  }

E 實驗內文實驗內文實驗內文實驗內文實驗內文實驗內文實驗內文實驗內文實驗內文實驗內文實驗內文實驗內文實驗內文實驗內文實驗內文實驗內文實驗內文實驗內文實驗內文實驗內文實驗內文實驗內文實驗內文實驗內文實驗內文實驗內文實驗內文實驗內文實驗內文實驗內文

\gasettingstable{實驗 E3 之基因演算法參數配置}
  {tbl:settings-of-experiment-results-trunk-gen-iii}
  { $1$ & $200$ & $100$ \\ }
  {
    阻攔點       & $1.00$ & \\
    支援點       & $0.75$ & \\
    巡邏點       & $0.50$ & \\
    遊戲物件數量 & $1.00$ & $Limit: [3, 5]$ \\
  }

\subsubsection{實驗 F - 個體數量的影響}
\label{sssec:experiment-results-trunk-idv}

F 實驗內文實驗內文實驗內文實驗內文實驗內文實驗內文實驗內文實驗內文實驗內文實驗內文實驗內文實驗內文實驗內文實驗內文實驗內文實驗內文實驗內文實驗內文實驗內文實驗內文實驗內文實驗內文實驗內文實驗內文實驗內文實驗內文實驗內文實驗內文實驗內文實驗內文

\gasettingstable{實驗 F1 之基因演算法參數配置}
  {tbl:settings-of-experiment-results-trunk-idv-i}
  { $1$ & $100$ & $50$ \\ }
  {
    阻攔點       & $1.00$ & \\
    支援點       & $0.75$ & \\
    巡邏點       & $0.50$ & \\
    遊戲物件數量 & $1.00$ & $Limit: [3, 5]$ \\
  }

F 實驗內文實驗內文實驗內文實驗內文實驗內文實驗內文實驗內文實驗內文實驗內文實驗內文實驗內文實驗內文實驗內文實驗內文實驗內文實驗內文實驗內文實驗內文實驗內文實驗內文實驗內文實驗內文實驗內文實驗內文實驗內文實驗內文實驗內文實驗內文實驗內文實驗內文

\gasettingstable{實驗 F2 之基因演算法參數配置}
  {tbl:settings-of-experiment-results-trunk-idv-ii}
  { $1$ & $100$ & $100$ \\ }
  {
    阻攔點       & $1.00$ & \\
    支援點       & $0.75$ & \\
    巡邏點       & $0.50$ & \\
    遊戲物件數量 & $1.00$ & $Limit: [3, 5]$ \\
  }

F 實驗內文實驗內文實驗內文實驗內文實驗內文實驗內文實驗內文實驗內文實驗內文實驗內文實驗內文實驗內文實驗內文實驗內文實驗內文實驗內文實驗內文實驗內文實驗內文實驗內文實驗內文實驗內文實驗內文實驗內文實驗內文實驗內文實驗內文實驗內文實驗內文實驗內文

\gasettingstable{實驗 F3 之基因演算法參數配置}
  {tbl:settings-of-experiment-results-trunk-idv-iii}
  { $1$ & $100$ & $200$ \\ }
  {
    阻攔點       & $1.00$ & \\
    支援點       & $0.75$ & \\
    巡邏點       & $0.50$ & \\
    遊戲物件數量 & $1.00$ & $Limit: [3, 5]$ \\
  }

\subsubsection{小結}
\label{sssec:experiment-results-trunk-summary}

實驗小結實驗小結實驗小結實驗小結實驗小結實驗小結實驗小結實驗小結實驗小結實驗小結實驗小結實驗小結實驗小結實驗小結實驗小結實驗小結





\section{標準化結果之比較}
\label{sec:experiment-normalized}

在第~\ref{ssec:method-segments-multiobjectives} 小節中,提出了方程式~\ref{eq:fitnesses-all} 的標準化效果會根據常數 $c$ 的大小,影響適應值前期與後期的成長幅度。在本小節中,將改變常數 $c$($c = 1$、$c = 2$、$c = 3$ 與 $c = 4$),觀察應用在「單一指標型」與「複合指標型」房間容器的收斂情形。單一指標型意指僅採用一項適應性函數(不包含平衡適應性函數);反之,複合指標型表示採用多項適應性函數。

\subsection{單一指標型 - 寶藏房}
\label{ssec:experiment-normalized-treasure}

\gasettingstable{實驗 X1 之基因演算法參數配置}
  {tbl:settings-of-experiment-normalized-treasure}
  { $1$ & $100$ & $200$ \\ }
  {
    守衛點       & $1.00$ & \\
    遊戲物件數量 & $1.00$ & $Limit: [2, 5]$ \\
  }




\subsection{複合指標型 - 戰鬥通道(狹路驅逐)}
\label{ssec:experiment-normalized-narrow}

\gasettingstable{實驗 Y1 之基因演算法參數配置}
  {tbl:settings-of-experiment-normalized-narrow}
  { $1$ & $100$ & $200$ \\ }
  {
    阻攔點       & $1.00$ & \\
    攔截點       & $0.75$ & \\
    陷阱點       & $0.75$ & \\
    遊戲物件數量 & $1.00$ & $Limit: [2, 3]$ \\
  }





\section{房型規模之比較}
\label{sec:experiment-density}

房型的大小較有可能直接影響和可行走瓦磚之數量。本階段的實驗中,我們提取第~\ref{sec:experiment-datacollection} 節的資料,將空白、敵人兩種類型的數量關係繪製成熱圖進行觀察。這是因為各項適應性函數在設計時,多以「敵人」與其餘敵人、其它遊戲物件或玩家動線為考量參考,因而推估二者間勢必存在者某些關係。





