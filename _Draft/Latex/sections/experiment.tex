\chapter{實驗結果與分析}
\label{cha:experiment}

本章節將探討前述章節之方法是否符合研究目標,將進行以下實驗:第~\ref{sec:experiment-results}小節中,針對所挑選的房間採用不同權重的適應性函數,會如何影響房間內的遊戲物件之配置結果。第~\ref{sec:experiment-yyy} 小節中,觀察染色體中的基因數量對於演化過程之影響。

\section{實驗定義}
\label{sec:experiment-definition}

我們使用 Invector 公司開發的 \textit{Third Person Controller - Melee Combat Template} 套件(以下簡稱 \textit{3rdPC})進行遊戲物件的設置。\textit{3rdPC} 提供遊戲開發者快速建構玩家控制角色 (Player Character) 、非玩家控制角色 (Non-Player Character) 與其角色控制器或人工智能,並可製作任何類型的第三人稱動作冒險遊戲或角色扮演遊戲。第~\ref{ssec:experiment-gameplaymanual} 小節中,說明 \textit{3rdPC} 提供的玩家角色的操作使用說明;第~\ref{ssec:experiment-gameobjects} 小節中,解釋實驗中的採用遊戲物件,其象徵意義與玩家角色互動方式;第~\ref{ssec:experiment-parameters} 小節中,將定義屬實驗變因的相關參數設定,包含任務語法與基因演算法。

\subsection{遊戲操作說明}
\label{ssec:experiment-gameplaymanual}

玩家所控制的角色能夠移動、攻擊、防禦、跳躍與蹲下等動作。移動使用方向鍵,角色朝前後左右與其斜向組合鍵的方向移動,過程中按下 shift 按鍵將會切換至奔跑模式;攻擊使用滑鼠左鍵單擊,連續使用能夠進行三階段的攻擊,每次攻擊將消耗耐力值;防禦使用滑鼠右鍵單擊,能夠減少受到敵方攻擊的損傷;跳躍按鍵鍵盤 space 按鍵,角色能夠進行原地跳躍,在移動或奔跑的過程中亦可使用;蹲下按下鍵盤 c 按鍵,角色能夠緩慢的移動,適合觀察敵方戰況的同時進行移動。

\subsection{遊戲物件說明}
\label{ssec:experiment-gameobjects}

(編輯中)

\subsubsection{敵人}
\label{sssec:experiment-gameobjects-enemy}

敵人 (編輯中)。

\subsubsection{寶箱}
\label{sssec:experiment-gameobjects-treasure}

當玩家靠近並朝向寶箱時,按下鍵盤 E 按鍵便可以開啟寶箱,取得寶箱內容物。

\subsubsection{陷阱}
\label{sssec:experiment-gameobjects-trap}

當玩家碰觸到陷阱時,會受到一定程度的損傷。

\subsection{實驗參數設定}
\label{ssec:experiment-parameters}

在任務語法階段時,我們將非合法 ... (編輯中)

地圖片段演化階段時,每一次世代的演化過程,有 80\% 機率父母代間會進行兩點交配 (two-point crossover);10\% 機率衍生子代會進行突變,染色體個體中有 5\% 至 20\% 的基因數量會轉換成其它的物件種類。

\section{資料收集}
\label{sec:experiment-datacollection}

在資料收集階段中,將收集第~\ref{sec:method-segments} 節中,房間容器進行基因演算法演化時,於實驗過程中輸出 CSV 格式資料。表~\ref{tbl:structure-of-rawdata-scores} 紀錄各回合、世代中,其各個個體(單一染色體)的適應值的得分狀況;表~\ref{tbl:structure-of-rawdata-positions} 紀錄各回合、世代中,其各個個體(單一染色體)的所有座標資訊與其遊戲物件類型。更多的原始資料節錄內容可參見附錄資源。

\begin{table}[ht]
  \centering
  \caption{演化適應值資料節錄}
  \label{tbl:structure-of-rawdata-scores}
  \bigskip
  \begin{tabular}{| c | c | c | l | l |}
    \hline
    \multicolumn{1}{ |c| }{回合編號} & \multicolumn{1}{ |c| }{世代編號} & \multicolumn{1}{ |c| }{染色體編號} & \multicolumn{1}{ |c| }{指標} & \multicolumn{1}{ |c| }{得分} \\\hline
    1 & 1 & 1 & Block     & 0 \\
    1 & 1 & 1 & Intercept & 0 \\
    1 & 1 & 1 & Patrol    & 0 \\
    1 & 1 & 1 & Guard     & 0 \\
    1 & 1 & 1 & Support   & 0 \\
    1 & 1 & 1 & Block     & 0 \\
    1 & 1 & 1 & Intercept & 0 \\
    \hline
  \end{tabular}
\end{table}

\begin{table}[ht]
  \centering
  \caption{演化座標資料節錄}
  \label{tbl:structure-of-rawdata-positions}
  \bigskip
  \begin{tabular}{| c | c | c | l | l |}
    \hline
    \multicolumn{1}{ |c| }{回合編號} & \multicolumn{1}{ |c| }{世代編號} & \multicolumn{1}{ |c| }{染色體編號} & \multicolumn{1}{ |c| }{座標} & \multicolumn{1}{ |c| }{遊戲物件類型} \\\hline
    1 & 1 & 1 & (1.0, 1.0, 0.0) & Empty \\
    1 & 1 & 1 & (0.0, 1.0, 0.0) & Empty \\
    1 & 1 & 1 & (0.0, 1.0, 1.0) & Empty \\
    1 & 1 & 1 & (0.0, 1.0, 2.0) & Empty \\
    1 & 1 & 1 & (0.0, 1.0, 3.0) & Empty \\
    \hline
  \end{tabular}
\end{table}

\section{演化結果與其品質}
\label{sec:experiment-results}

在第~\ref{ssec:method-segments-appliedonvolumes} 小節中,展示了寶藏房與戰鬥通道(狹路驅逐、鎮守要道)三種空間的局部佈局演化結果,隨著房間容器搭配不同的適應性函數,便能夠生成出多樣性的遊戲物件佈局。在本小節中,將針對上述三種房間容器的演化結果做更進一步的分析。

\subsection{寶藏房}
\label{ssec:experiment-results-treasure}

\subsubsection{實驗 A - 世代數量的影響}
\label{sssec:experiment-results-treasure-i}

\begin{table}[ht]
  \centering
  \caption{實驗 A1 之基因演算法基本參數配置}
  \label{tbl:settings-of-experiment-results-treasure-i}
  \bigskip
  \begin{tabular}{| c | c | c |}
    \hline
    \multicolumn{1}{ |c| }{回合次數} & \multicolumn{1}{ |c| }{世代數量} & \multicolumn{1}{ |c| }{個體數量} \\\hline
    $1$ & $100$ & $200$ \\
    \hline
  \end{tabular}
\end{table}

\begin{table}[ht]
  \centering
  \caption{實驗 A1 之基因演算法適應性函數權重配置}
  \label{tbl:fitness-of-experiment-results-treasure-i}
  \bigskip
  \begin{tabular}{| c | c | c |}
    \hline
    \multicolumn{1}{ |c| }{指標} & \multicolumn{1}{ |c| }{權重} & \multicolumn{1}{ |c| }{備注} \\\hline
    守衛點       & $1.00$ & \\
    遊戲物件數量 & $1.00$ & $Limit: [2, 5]$ \\
    \hline
  \end{tabular}
\end{table}

\subsubsection{實驗 B - PPPPP}
\label{sssec:experiment-results-treasure-ii}

\begin{table}[ht]
  \centering
  \caption{實驗 B 之基因演算法基本參數配置}
  \label{tbl:settings-of-experiment-results-treasure-ii}
  \bigskip
  \begin{tabular}{| c | c | c |}
    \hline
    \multicolumn{1}{ |c| }{回合次數} & \multicolumn{1}{ |c| }{世代數量} & \multicolumn{1}{ |c| }{個體數量} \\\hline
    $1$ & $x$ & $y$ \\
    \hline
  \end{tabular}
\end{table}

\subsection{戰鬥通道(狹路驅逐)}
\label{ssec:experiment-results-narrow}

編輯中。

\subsection{戰鬥通道(鎮守要道)}
\label{ssec:experiment-results-trunk}

編輯中。

\section{標準化結果之比較}
\label{sec:experiment-normalized}

在第~\ref{ssec:method-segments-multiobjectives} 小節中,提出了方程式~\ref{eq:fitnesses-all} 的標準化效果會根據常數 $c$ 的大小,影響適應值前期與後期的成長幅度。在本小節中,將改變常數 $c$($c = 1$、$c = 2$、$c = 3$ 與 $c = 4$),觀察應用在「單一指標型」與「複合指標型」房間容器的收斂情形。單一指標型意指僅採用一項適應性函數(不包含平衡適應性函數);反之,複合指標型表示採用多項適應性函數。

\subsection{單一指標型 - 寶藏房}
\label{ssec:experiment-normalized-treasure}

編輯中。

\subsection{複合指標型 - 戰鬥通道(狹路驅逐)}
\label{ssec:experiment-normalized-narrow}

編輯中。

\section{房型規模之比較}
\label{sec:experiment-density}

房型的大小較有可能直接影響和可行走瓦磚之數量。本階段的實驗中,我們提取第~\ref{sec:experiment-datacollection} 節的資料,將空白、敵人兩種類型的數量關係繪製成熱圖進行觀察。這是因為各項適應性函數在設計時,多以「敵人」與其餘敵人、其它遊戲物件或玩家動線為考量參考,因而推估二者間勢必存在者某些關係。

