\renewcommand\thetable{\arabic{chapter}-\arabic{table}}
%\renewcommand\thefigure{\arabic{chapter}-\arabic{figure}} 
\chapter{實驗結果與分析}
\label{cha:experiment}

%本研究成果為設計程序化生成遊玩特徵的遊戲關卡,
本章節將探討前述章節之方法是否符合研究目標,將進行以下實驗:~\ref{sec:experiment-results}小節中,針對所挑選的房間採用不同權重的適應性函數,會如何影響房間內的遊戲物件之配置結果。~\ref{sec:experiment-yyy} 小節中,觀察染色體中的基因數量對於演化過程之影響。~\ref{sec:experiment-zzzz} 小節,進行評估不同的交配策略下,何者優劣優勢之情形。

\section{實驗定義}
\label{sec:experiment-definition}

我們使用 Invector 公司開發的 \textit{Third Person Controller - Melee Combat Template} 套件(以下簡稱 \textbf{3rdPC})進行遊戲物件的設置。\textbf{3rdPC} 提供遊戲開發者快速建構玩家控制角色 (Player Character) 、非玩家控制角色 (Non-Player Character) 與其角色控制器或人工智能,並可製作任何類型的第三人稱動作冒險遊戲或角色扮演遊戲。

\subsection{遊戲操作說明}
\label{ssec:experiment-gameplaymanual}

玩家所控制的角色能夠移動、攻擊、防禦、跳躍與蹲下等動作。移動使用方向鍵,角色朝前後左右與其斜向組合鍵的方向移動,過程中按下 shift 按鍵將會切換至奔跑模式;攻擊使用滑鼠左鍵單擊,連續使用能夠進行三階段的攻擊,每次攻擊將消耗耐力值;防禦使用滑鼠右鍵單擊,能夠減少受到敵方攻擊的損傷;跳躍按鍵鍵盤 space 按鍵,角色能夠進行原地跳躍,在移動或奔跑的過程中亦可使用;蹲下按下鍵盤 c 按鍵,角色能夠緩慢的移動,適合觀察敵方戰況的同時進行移動。

\subsection{遊戲物件說明}
\label{ssec:experiment-gameobjects}

(編輯中)

\subsubsection{敵人}
\label{sssec:experiment-gameobjects-enemy}

敵人 (編輯中)。

\subsubsection{寶箱}
\label{sssec:experiment-gameobjects-treasure}

當玩家靠近並朝向寶箱時,按下鍵盤 E 按鍵便可以開啟寶箱,取得寶箱內容物。

\subsubsection{陷阱}
\label{sssec:experiment-gameobjects-trap}

當玩家碰觸到陷阱時,會受到一定程度的損傷。

\subsection{基因演算法參數定義}
\label{ssec:experiment-gaparameters}

在任務語法階段時,我們將非合法 ... (編輯中)

地圖片段演化階段時,每一次世代的演化過程,有 80\% 機率父母代間會進行兩點交配 (two-point crossover);10\% 機率衍生子代會進行突變,染色體個體中有 5\% 至 20\% 的基因數量會轉換成其它的物件種類。

\section{資料收集}
\label{sec:experiment-datacollection}

在資料收集階段中,將收集第~\ref{sec:method-segments} 節中,房間容器進行基因演算法時,於實驗中各回合、世代與其個體(單一染色體)中的得分狀況。

\begin{table}[ht]
  \centering
  \caption{原始資料之欄位示意}
  \label{tbl:structure-of-rawdata}
  \bigskip
  \begin{tabular}{| l | l | l | l | l | l | l | l |}
    \hline
    回合編號 & 世代編號 & 染色體編號 & 指標 & 得分 & 座標 & 類型 & 房間 \\
    1 & 1 & 1 & Block     & 0 & (12.0, 1.0, 0.0)  & Empty & Room A \\
    1 & 1 & 1 & Intercept & 0 & (12.0, 1.0, 0.0)  & Empty & Room A \\
    1 & 1 & 1 & Patrol    & 0 & (12.0, 1.0, 0.0)  & Empty & Room A \\
    1 & 1 & 1 & Guard     & 0 & (12.0, 1.0, 0.0)  & Empty & Room A \\
    1 & 1 & 1 & Support   & 0 & (12.0, 1.0, 0.0)  & Empty & Room A \\
    1 & 1 & 1 & Block     & 0 & (24.0, 1.0, 12.0) & Empty & Room A \\
    1 & 1 & 1 & Intercept & 0 & (24.0, 1.0, 12.0) & Empty & Room A \\
    \hline
  \end{tabular}
\end{table}

\section{演化結果與其品質}
\label{sec:experiment-results}

content here.

\section{各世代演化狀態}
\label{sec:experiment-evolutions}

由於各點基因,其中的空格、敵人兩種類型,於每一遊玩特徵的指標都具有高度意義與相關性。

\section{標準化結果之比較}
\label{sec:experiment-normalized}

content here.

\section{房型規模之比較}
\label{sec:experiment-yyy}

房型的大小較有可能直接影響和可行走瓦磚之數量。本階段的實驗中,我們提取~\ref{sec:experiment-datacollection} 節的資料,將空白、敵人兩種類型的數量關係繪製成熱圖進行觀察。這是因為各項適應性函數在設計時,多以「敵人」與其餘敵人、其它遊戲物件或玩家動線為考量參考,因而推估二者間勢必存在者某些關係。

