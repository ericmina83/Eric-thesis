%\renewcommand\thefigure{\arabic{chapter}-\arabic{figure}}
%\bibliographystyle{unsrt} 
\chapter{緒論}
\label{cha:intro}

程序化內容自動生成 (Procedural Content Generation) 在過去就廣泛被應用於遊戲設計領域,其主要目的為增加遊戲內容的隨機性與多樣性。在本文中,我們針對遊戲過程中的遊玩特徵 (gameplay patterns) 進行抽象化,使用程序化生成技術產生帶有意義遊戲關卡內容,藉此消彌或降低因隨機性所產生的不穩定要素,以改善並豐富遊戲體驗。

我們將遊戲關卡的構成劃分為任務 (Missions) 與空間 (Space) 兩種結構後,空間會依照任務結構進行有意義的轉換,接著依照遊玩特徵定義基因演算法 (Genetic Algorithms) 的演化依據。讓玩家在進行遊戲時能夠遵循關卡設計師的劇情脈絡外,亦能夠體驗到有意義且多樣化的遊戲關卡內容。

\section{迷宮探索遊戲 (dungeon crawl) 類型介紹}

content here.

\subsection{迷宮探索遊戲的歷史}

content here.

\section{研究動機}

content here.

迷宮探索

\section{研究目的}

content here.

\section{預計研究貢獻}

content here.

\section{本論文之章節結構}

content here.


