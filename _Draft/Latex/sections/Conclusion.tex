\chapter{結論與後續工作}
\label{cha:conclusions}

\section{貢獻與結論}

在實驗結果的章節中,房間容器依照遊玩特徵指標進行演化的結果,即使透過抽象化(體素化)空間設計其遊玩特徵指標,驗證即使精確度降低的情形下,確保結果具有一定品質與遊戲體驗。且提出之適應值標準化方法能夠有效解決多目標最佳化問題 (Multi-objective optimization),讓不同的適應值能相互均衡發展與演化。綜合所有方法所彙整的關卡自動生成工具~\textit{Dungeon Generator},提供了完整流程的關卡生成解決方案。

% 回顧第~\ref{cha:intro} 章提及之研究目標,其一、「存在多種不同面向的適應性函數,要如何進行有效的數值標準化?」;其二、「人為調整相關基因演算法演化參數時,對於關卡生成會有什麼影響?」

藉由~\textit{Dungeon Generator} 高度抽象遊戲概念的遊戲開發輔助工具,主以視覺化其概念邏輯之結構,輔以高度語意化的標籤進行數值微調,大幅降低關卡設計師的設計成本,終以加速關卡的生成產量。且~\textit{Dungeon Generator} 藉由有意義為核心目標進行關卡生成,消彌了市面上絕大多數關卡生成工具的隨機不確定性,便能夠控管玩家的遊玩體驗。

\section{限制與未來研究方向} 

在設計適應性函數的權重時,關卡設計師仍需要對於預期結果有一定的知識與掌握度,才能夠正確的產出預期的遊玩特徵。儘管增加了關卡設計師的工具使用門檻,若能熟悉本論文所提出的方法與工具相信對於未來遊戲開發上必有相當大的幫助。

此外,若能夠將遊玩的歷程加入節奏 (rhythm),隨著遊戲進程調整遊戲的難易度,令前述適應性函數權重進行動態調整,方能夠帶來更加不同的遊戲體驗。更進一步地導入機器學習技術,讓程序學習良好的關卡配置,建立其數學模型以自動化規劃遊戲特徵的權重。